\documentclass[useAMS,usenatbib]{mn2e}
\usepackage{graphicx}
\usepackage{color}
%\usepackage{amstex, amssymb}
\voffset-1.25cm

% If your system does not have the AMS fonts version 2.0 installed, then
% remove the useAMS option.
%
% useAMS allows you to obtain upright Greek characters.
% e.g. \umu, \upi etc.  See the section on "Upright Greek characters" in
% this guide for further information.
%
% If you are using AMS 2.0 fonts, bold math letters/symbols are available
% at a larger range of sizes for NFSS release 1 and 2 (using \boldmath or
% preferably \bmath).
%
% The usenatbib command allows the use of Patrick Daly's natbib.sty for
% cross-referencing.
%
% If you wish to typeset the paper in Times font (if you do not have the
% PostScript Type 1 Computer Modern fonts you will need to do this to get
% smoother fonts in a PDF file) then uncomment the next line
% \usepackage{Times}

%%%%% AUTHORS - PLACE YOUR OWN MACROS HERE %%%%%


\newcommand{\Rq}{$\rm R_{1/4} $}

\newcommand{\hMsun}{$h^{-1}\rm{M_{\odot}}$}
\newcommand{\hMpc}{$h^{-1}\rm{Mpc}$}
\newcommand{\hGpc}{$h^{-1}\rm{Gpc}$}


\newcommand{\etal}{et al.}
\newcommand{\rhocrit}{\rho_{\rm c}}
\newcommand{\rhorms}{\rho_{rms}}
\newcommand{\xoff}{x_{off}}
\newcommand{\cvir}{c_{vir}}
\newcommand{\Rvir}{R_{\rm vir}}
\newcommand{\Mvir}{M_{\rm vir}}
\newcommand{\rs}{r_s}
\newcommand{\Jvir}{J_{\rm vir}}
\newcommand{\Vvir}{V_{\rm vir}}
\newcommand{\Nvir}{N_{\rm vir}}
\newcommand{\Ms}{M_{s}}
\newcommand{\Mth}{$\mathrm{M}_\mathrm{th}$}
\newcommand{\Mdom}{$\mathrm{M}_\mathrm{dom}$}

\newcommand{\mnras}{MNRAS}
\newcommand{\apj}{ApJ}
\newcommand{\apjs}{ApJs}
\newcommand{\apjl}{ApJL}
\newcommand{\aj}{AJ}
\newcommand{\aap}{A\&A}
\newcommand{\prd}{Phys. Rev. D.}
\newcommand{\physrep}{Phy. Rep.}


%%%%%%%%%%%%%%%%%%%%%%%%%%%%%%%%%%%%%%%%%%%%%%%%

\title[Power Spectrum Comparison...]
{Power Spectrum Comparison... y ya veremos que mas!}
\author[O. Nicol\'as Gomez-Giraldo, Juan C. Mu\~noz-Cuartas {\bf el orden habra de cambiar, el primer autor se lo merece}]
       {O. Nicol\'as Gomez-Giraldo $^{1}$\thanks{Email : onicolas.gomez@udea.edu.co};
         Juan C. Mu\~noz-Cuartas $^{1}$\thanks{Email : jcmunozc@udea.edu.co};\\ $^{1}$ FACom
         - Instituto de F\'isica, FCEN, Universidad de Antioquia UdeA, Calle
         70 No. 52-21, Medell\'in, Colombia.\\ 
         }
\begin{document}

\date{Accepted XXXX December XX. Received XXXX December XX; in original form
  2015 Jan 30}

\pagerange{\pageref{firstpage}--\pageref{lastpage}} \pubyear{2002}

\maketitle

\label{firstpage}

\begin{abstract}

Here we put the abstract

\end{abstract}

\begin{keywords}

\end{keywords}


%%%%%%%%%%%%%%%%%%%%%%%%%%%%%%%%%%%%%%%%%%%%%%%%%%%%%%%%%%%%%%%%%%%%%%
%% SECTION 1: Introduction
%%%%%%%%%%%%%%%%%%%%%%%%%%%%%%%%%%%%%%%%%%%%%%%%%%%%%%%%%%%%%%%%%%%%%%
\section{Introduction}

Presentacion del problema
Resumen de la bibliografia
que es lo que hacemos
por que es importante
esquleto del paper

Objetivo del paper: Mostrar las bondades de la estimacion del
bispectrum usando deaubichis. En particular mostrando los efectos del
aliasing en el bispectrum.




%%%%%%%%%%%%%%%%%%%%%%%%%%%%%%%%%%%%%%%%%%%%%%%%%%%%%%%%%%%%%%%%%%%%%%
%% SECTION 2: OBSERVATIONAL DATA
%%%%%%%%%%%%%%%%%%%%%%%%%%%%%%%%%%%%%%%%%%%%%%%%%%%%%%%%%%%%%%%%%%%%%%
\section{Theory: Power spectrum, bispectrum and estimation} 
\label{sec:theory}

Teoria basica del Pk y Bk. Definiciones, estimadores (formalismo)

\subsection{Mass assignment scheme: NGP, CIC, TGP, Deaubichis}
\label{sec:theory:MAS}

Teoria basica del MAS. Definiciones, estimadores
(formalismo). Implicaciones ventajas y desventajas de cada metodo.

(Figura: figurita con las formas!)

\subsection{Aliasing: NGP, CIC, TGP, Deaubichis}
\label{sec:theory:aliasing}

Teoria basica del aliasing. Definiciones, estimadores
(formalismo). Implicaciones ventajas y desventajas de cada metodo.
presentar Calculo del aliasing del aliasing general (nuevo?)

(Figura: Podemos hacer una grafica del formalismo ``general" del
aliasing comparando con lo que hay en el mercado)

\section{Methods}
\label{sec:methods}

\subsection{Simulations}
\label{sec:methods:simulations}

Describir los datos de las simulaciones, por que? pa' que?

\subsection{Power spectrum and bispectrum estimators}
\label{sec:methods:pkbk}

Como se estima el espectro de potencia y bispectrum en los datos?
detalles de la implementacion.
Pruebas? Todo funciona bien? -> comparaciones con el modelo teorico!

Una buena descripcion para dummies de como calcular el bispectrum.

%%%%%%%%%%%%%%%%%%%%%%%%%%%%%%%%%%%%%%%%%%%%%%%%%%%%%%%%%%%%%%%%%%%%%%
%% SECTION 3: RESULTS
%%%%%%%%%%%%%%%%%%%%%%%%%%%%%%%%%%%%%%%%%%%%%%%%%%%%%%%%%%%%%%%%%%%%%%
\section{Results} 
\label{sec:results}

\subsection{Power spectrum}
\label{sec:results:pk}

Discutir el resultado de Pk con los diferentes MAS (Colombi
et.al. 2009). Efectos del aliasing, shot noise y demas.
Como va eso en el Bk?

(figura: Pk con los diferentes MAS, mascas de kf, etc)

\subsection{Bispectrum}
\label{sec:results:bk}

Asi nos dio el bispectrum
comparacion del bispectrum con los diferentes MAS.
Analisis de diferencias

(figura: Bk con los diferentes MAS, mascas de kf, etc)

\subsection{Aliasing}
\label{sec:results:bk}

Estimar (cuantitativaente) los efectos del aliasing
(Figura: estimacion del aliasing, variaciones fraccionales, etc)


\section{Summary and discussion}



\section*{Acknowledgments}


\bibliography{references}
% \begin{thebibliography}{99}
%  
% % \bibitem[{Abazajian}, K.~N. \& {Adelman-McCarthy}, J.~K. \& {Ag{\"u}eros}, M.~A. \& 
% % 	{Allam}, S.~S. \& {Allende Prieto}, C. \& {An}, D. \& {Anderson}, K.~S.~J. \& 
% % 	{Anderson}, S.~F. \& {Annis}, J. \& {Bahcall}, N.~A. \& et al.]{2009ApJS..182..543A} Abazajian, K.~N. and Adelman-McCarthy, J.~K. and Ag{\"u}eros, M.~A. and 
% % 	Allam, S.~S. and Allende Prieto, C. and An, D. and Anderson, K.~S.~J. and 
% % 	Anderson, S.~F. and Annis, J. and Bahcall, N.~A. \& et al.\ 2009, \apjs, 543, 558  
% 
% \bibitem[Becker \& Kravtsov(2011)]{2011ApJ...740...25B} Becker, M.~R.,
%   \& Kravtsov, A.~V.\ 2011, \apj, 740, 25
% 
% \bibitem[Bertschinger(2001)]{2001ApJS..137....1B} Bertschinger,
%   E.\ 2001, \apjs, 137, 1
% 
% \bibitem[Betancort-Rijo et al.(2006)]{2006ApJ...649..579B} Betancort-Rijo, 
% J.~E., Sanchez-Conde, M.~A., Prada, F., 
% \& Patiri, S.~G.\ 2006, \apj, 649, 579 
% 
% \bibitem[Bryan \& Norman(1998)]{1998ApJ...495...80B} Bryan, G.~L., \&
%   Norman, M.~L.\ 1998, \apj, 495, 80
% 
% \bibitem[Gil-Mar{\'{\i}}n et al.(2011)]{2011MNRAS.414.1207G} 
% Gil-Mar{\'{\i}}n, H., Jimenez, R., \& Verde, L.\ 2011, \mnras, 414, 1207
% 
% \bibitem[Haas et al.(2012)]{2012MNRAS.419.2133H} Haas, M.~R., Schaye,
%   J., \& Jeeson-Daniel, A.\ 2012, \mnras, 419, 2133
% 
% \bibitem[Komatsu et al. (2009)]{2009ApJS..180..330K} Komatsu E., et
%   al., 2009, ApJS, 180, 330
% 
% \bibitem[Kowalski et al.(2008)]{2008ApJ...686..749K} Kowalski, M., Rubin, 
% D., Aldering, G., et al.\ 2008, \apj, 686, 749 
% 
% \bibitem[Macci{\`o} et al.(2008)]{2008MNRAS.391.1940M} Macci{\`o}, A.~V.,
%   Dutton, A.~A., \& van den Bosch, F.~C.\ 2008, \mnras, 391, 1940
% 
% \bibitem[Mandelbaum et al.(2006)]{2006MNRAS.372..758M} Mandelbaum, R., 
% Seljak, U., Cool, R.~J., et al.\ 2006, \mnras, 372, 758 
% 
% \bibitem[Mu{\~n}oz-Cuartas et al.(2011)]{2011MNRAS.411..584M} 
% Mu{\~n}oz-Cuartas, J.~C., Macci{\`o}, A.~V., Gottl{\"o}ber, S., 
% \& Dutton, A.~A.\ 2011a, \mnras, 411, 584 
% 
% \bibitem[Mu{\~n}oz-Cuartas et al.(2011)]{2011MNRAS.417.1303M}
%   Mu{\~n}oz-Cuartas, J.~C., M{\"u}ller, V., \& Forero-Romero,
%   J.~E.\ 2011b, \mnras, 417, 1303
% 
% \bibitem[Oguri \& Hamana(2011)]{2011MNRAS.414.1851O} Oguri, M., \&
%   Hamana, T.\ 2011, \mnras, 414, 1851
% 
% \bibitem[Pope et al.(2004)]{2004AIPC..743..120P} Pope, A., Szalay, A., 
% Matsubara, T., et al.\ 2004, The New Cosmology: Conference on Strings and 
% Cosmology, 743, 120
% 
% \bibitem[Prada et al.(2006)]{2006ApJ...645.1001P} Prada, F., Klypin, A.~A., 
% Simonneau, E., et al.\ 2006, \apj, 645, 1001 
% 
% \bibitem[Riess et al.(1998)]{1998AJ....116.1009R} Riess, A.~G.,
%   Filippenko, A.~V., Challis, P., et al.\ 1998a, \aj, 116, 1009
% 
% \bibitem[Riess(1998)]{1998AAS...192.1706R} Riess, A.~G.\ 1998b, Bulletin of 
% the American Astronomical Society, 30, 843
% 
% \bibitem[Spergel et al.(2007)]{2007ApJS..170..377S} Spergel, D.~N., et
%   al.\ 2007, \apjs, 170, 377
% 
% \bibitem[Stadel(2001)]{2001PhDT........21S} Stadel, J.~G.\ 2001,
%   Ph.D.~Thesis.
% 
% \bibitem[Tavio et al.(2008)]{2008arXiv0807.3027T} Tavio, H., Cuesta, A.~J., 
% Prada, F., Klypin, A.~A., \& Sanchez-Conde, M.~A.\ 2008, arXiv:0807.3027 
% 
% \bibitem[Tegmark et al.(2004)]{2004PhRvD..69j3501T} Tegmark, M., et
%   al.\ 2004, \prd, 69, 103501
% 
% \bibitem[Tinker et al.(2011)]{2011arXiv1107.5046T} Tinker, J., Wetzel, A., 
% \& Conroy, C.\ 2011, arXiv:1107.5046 
% 
% \bibitem[Wang et al.(2009)]{2009MNRAS.394..398W} Wang, H., Mo, H.~J.,
% Jing, Y.~P., Guo, Y., van den Bosch, F.~C., \& Yang, X.\ 2009, \mnras,
% 394, 398
% 
% \bibitem[Wechsler et al.(2002)]{2002ApJ...568...52W} Wechsler, R.~H., Bullock,
%   J.~S., Primack, J.~R., Kravtsov, A.~V., \& Dekel, A.\ 2002, \apj, 568, 52
% 
% \bibitem[White \& Rees(1978)]{1978MNRAS.183..341W} White, S.~D.~M., \&
%   Rees, M.~J.\ 1978, \mnras, 183, 341
% 
%   
% \end{thebibliography}

\bsp

\label{lastpage}

\end{document}
